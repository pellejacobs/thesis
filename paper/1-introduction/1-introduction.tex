\chapter{Introduction}

\textbf{THIS TEXT SHOULD BE UPDATED WHILE WRITING CHAPTERS}

% SOME GENERAL INTRODUCTION BS

Hitting a market capitalization of \$13.8 Billion in December 2013, it was clear that a once fringe cryptocurrency called ``Bitcoin'' had hit mainstream. By building on decades of research in decentralized currencies and cryptography, Bitcoin is the first successful implementation of a truly decentralized currency. Most of this success has been attributed to an innovation called ``Blockchain''. However, this success has opened up a deeper concept: the power of technology to replace a previously thought middleman.

% CENTRAL QUESTION

Now, the question is: ``How can this idea to replace a middleman with technology be applied to collaborative, distributed databases and what is the role of blockchain is this concept?'' The next three chapters provide an answer to this question:

% READING GUIDE

In the next chapter, three main concepts that are essential aspects of this question are discussed. First public key encryption is explained, as it is the basis for modern encryption and digital signatures. Next, the concept of the blockchain, its properties and some well-known implementation are examined. Finally, the concept the thesis delves into distributed databases, the difference between distributed, decentralized and centralized databases and some non-blockchain examples of distributed databases and networks.

The third chapter explains how this thesis tries to solve the research question in the final chapter.

The final chapter investigates the viability of specific proposals to answer the research question: how every proposal tries to solve its specific problem, the advantages and the disadvantages of its approach.