\chapter{Introduction}

% SOME GENERAL INTRODUCTION BS

Hitting a market capitalization of \$13.8 billion in December 2013, it was clear that a once fringe cryptocurrency called ``Bitcoin'' had hit mainstream. By building on decades of research in decentralized currencies and cryptography, Bitcoin is the first successful implementation of a truly decentralized currency. Most of this success has been attributed to an innovation called ``blockchain''. However, this success has opened up a deeper concept: the power of technology to replace a previously thought indispensable middleman.

% CENTRAL QUESTION

Now, the question is: ``How can the idea to replace a middleman with technology be applied to collaborative databases and how does this involve blockchain technology?'' The next three chapters provide an answer to this question.

% READING GUIDE

In the following chapter, three main concepts are discussed that are essential aspects of this question. First, public key encryption is explained, as it is the basis for modern encryption and digital signatures. Next, the concept of the blockchain, its properties, and some well-known implementations are examined. Finally, the thesis delves into distributed databases: the difference between distributed, decentralized and centralized databases, the CAP theorem, common issues with distributed databases and some examples of distributed databases and networks.

Next, the thesis explains how it tries to solve the research question in the final chapter. Collaborative databases will be split up into two groups: centralized databases, and distributed databases. Both groups have a middleman that could be replaced, but the role of this middleman differs greatly between both groups. 

The final chapter investigates the viability of specific proposals trying to remove the middleman. Every section discusses advantages and disadvantages, features and constraints of the proposal.
