\chapter{Introduction}

\textbf{THIS IS OLD AN NEEDS A REWRITE}

% SOME GENERAL INTRODUCTION BS

Hitting a market capitalization of \$13.8 Billion in December 2013, it was clear that a once fringe cryptocurrency called 'Bitcoin' had hit mainstream. By building on decades of research in decentralized currencies and cryptography, Bitcoin is the first successful implementation of a truly decentralized currency. Most of this success has been attributed to an innovation called 'Blockchain'.

% CENTRAL QUESTION

Now, the question is: 'How can this blockchain technology be leveraged to power collaborative, distributed databases?' To provide an answer to this question, this thesis will be split in three chapters:

% READING GUIDE

In the first chapter, the blockchain technology will be thoroughly discussed. The chapters starts out with a description of the concept 'blockchain', the specific characteristic of blockchains and differences between implementations of blockchains. The following part looks into why the blockchain was developed in the first place, looking at the advantages a blockchain provides compared to its alternatives. Finally, several implementations of blockchains and blockchain applications are reviewed.

The second chapter describes distributed and decentalised databases. To maintain consistency, it follows a similar structure as the first chapter: a definition of what is meant with the terms 'distributed' and 'decentralized' and how these concepts can be applied to databases. Next follows a comparison between distributed and decentralized databases and fully centralized databases, looking at the advantages and disadvantages of both implementations. Finally, some common implementations of distributed and decentralized database are reviewed.

The final chapter combines the previous two chapters by focusing on how to bring collaborative, distributed databases to a blockchain. A first section will discuss the general advantages but also the trade offs of bringing a database to the blockchain. The second section suggests several possible implementations.