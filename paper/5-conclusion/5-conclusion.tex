\chapter{Conclusion}

Throughout this paper, it has become clear several trusted third parties can be replaced. In the private environments, central storage providers and central interaction management providers can be replaced by blockchains. In the public environment, privileged professions such as notaries can be replaced by the more secure alternative of public key encryption and blockchains.

I think governments could largely reduce the amount signature disputes by incorporating blockchains into the law. Although this paper specifically focussed on testaments, a similar case could be made for other contracts. 

If the government does move to blockchains, several aspects have to be taken into account concerning which blockchains should be eligible. I am convinced the law should not limit itself to one single blockchain, but should rather put requirements on which public blockchains could be used: which consensus algorithm should be used, what is the minimum hashing power the blockchain should posses and what is the maximum centralization of the hashing power. 

Personally, I am convinced only proof of work has currently proven to be extremely secure. Concerning minimum hashing power, I would recommend to make the law time agnostic by defining the minimum hashing power as a percentage of the estimated total computing power currently available in the world. Concerning centralization, I would recommend to put maximum cumulative shares on the largest miners. For example, the largest miner should not own more than 30\% of total hashing power, the largest two miners should not own more than 40\% combined and the largest three miners should not control more than 50\%. 

But even without laws specifically mentioning blockchain, I think blockchain verified testaments (and contracts) are already more secure. As mentioned before, a notary is not required for a Belgian testament to be valid. It is therefore only a matter of time before a case is brought before a court concerning which testament is valid: a digitally signed, blockchain verified document or a notary witnessed one. This will set a precedent for the future, probably forcing the legislative branch to incorporate blockchains into the laws.

