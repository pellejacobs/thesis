\chapter{What is the blockchain and what can it be used for}

\section{Definition}
In the last few years, the blockchain as a concept has been given a wide range in meanings and definitions. Therefore, we will start out this paper by thoroughly defining this concept, to avoid any misunderstanding or confusion further on.

Antonopoulos (2014) \cite{antonopoulos:2014} probably defines it best: ``The blockchain data structure is an ordered, back-linked list of blocks and transactions.'' (p. 159). There are several aspects in this definition that need further explanation.

\subsection{Back-linked lists as data structure}

First of all, we are talking about a data structure: ``A specialised format for organising and storing data'' \cite{data-structure}. There are several types of data structures such as arrays, lists, graphs, trees, etc.

Next, a blockchain is a specialised version of an ``ordered, back-linked list''. A list is data structure that combines a number of ordered values (Abelson 1996 \cite{abelson:1996}).  Incidentally, the ``ordered'' in the definition provided by Antonopoulos is superfluous, as by definition every list is ordered. An unordered list would no longer be a list, but would be a set instead. A blockchain will use back-linking to preserve the order of the values. This means that all values have a reference to the previous value in the list. For a blockchain, this means that every block (see subsection \ref{subsec:Block-s}) in the blockchain contains the ``hash'' of the previous block. We will go further into hashes in a next section.

\subsection{Transactions}

Antonopoulos (2014) \cite{antonopoulos:2014} defines transactions as: ``data structures that encode the transfer of value between participants in the bitcoin system'' (p. 109). Although it gives a good idea of what transactions are, even in the context of the bitcoin blockchain is this definition not completely correct: despite most transactions being value transactions between participants, a transaction can also store data on a blockchain. Even the bitcoin blockchain allows non-monetary transactions to be included.

Therefore, we can conclude there are two types of transactions: monetary transactions and non-monetary transactions. In a monetary transaction, there is a transfer of value (eg. bitcoin) between participants. It must be noted that this does not have to be an instant transfer. Several blockchain protocols allow for a custom conditions to be scripted into the transaction.

On the other side, there are non-monetary transactions that store data onto the blockchain. An example of a common used non-monetary transaction is the encoding of a document on the blockchain. The specific advantages of such practices will be discussed further in section \ref{sec:Advantages of the blockchain} ``Advantages of the blockchain''.

Transactions are the raison d'\^{e}tre of the blockchain. All aspects that we will discuss in this chapter exist to make sure that transactions can be validated, propagated over the network and added to a distributed ledger.

\subsection{Blocks}
\label{subsec:Blocks}

Antonopoulos (2014) \cite{antonopoulos:2014} defines a block as: ``a container data structure that aggregates transactions for inclusion in the public ledger, the blockchain.'' (p. 160). A block is literally a wrapper for multiple transactions to make it possible for them to be included into the blockchain. These blocks then form the ordered values of the back-linked list addressed in the beginning of this section, each linking to the previous block in the list.

To conclude, the blockchain is a list of blocks, with each block containing several transactions.

\newpage
\section{Advantages of the blockchain}
\label{sec:Advantages of the blockchain}
