\section{Distributed cloud storage}
\label{blockchain-distributed-storage}

\iffalse

- idea: instead of trusting a central party: 
  - divide your storage data into little pieces and send the pieces to different storage providers => process of dividing is referred to as sharding. storage providers can be anyone: large companies with data centers or individuals with available space on their hard drive. 
  - you send your shards of data to a storage provider. you can send every shard just once, or you could store it multiple times to be sure your data never disappears. the renter pays the hosts for the time they store their data.

- however: multiple issues arise in this system: what if a host promises to store your data, but actually does not, what if a renter doesn't pay for the used storage space?

- there are currently several solutions out there that try implement this concept, such as siacoin, storj and swarm. Others, such as ipfs and maidsafe, want to go even further trying to replace the internet as we know it.

\fi

A distributed cloud storage network replaces a trusted storage provider such as Amazon or Dropbox with a large network of storage providers. Storage providers can be large companies with data centers to their disposal or individuals with free storage space on their hard drive. These storage providers are referred to as the hosts. 

When a user, called a renter, wants to store his data on the network, he divides his data into small pieces. The renter then sends these pieces to the hosts. For extra redundancy and security, the renter duplicates his pieces several times on the network. The renter then pays the hosts for the time they store their data.

This simple concept faces multiple issues. A host could promise to store a renter's data without actually keeping the data. A renter would have to do some insane redundancy duplication on the network to safely store his data, while still being at risk of losing an essential piece. On the other hand, a host storing a renter's data is never guaranteed the renter will actually pay.

Currently, several solutions try to implement this concept. Siacoin \cite{siacoin} and Storj \cite{storj} focus on cloud storage. Others, such as IPFS \cite{IPFS} and Maidsafe \cite{maidsafe}, want to go further and are trying to replace the internet as it currently works. This thesis will focus on cloud storage and compare two current commercial solutions: Siacoin and Storj.


\subsection{Storj}

\iffalse
# storj

- storj: uses a pay as you go system. every now and then you ask your storage providers for a proof that they are still holding your data. if you receive this proof, you pay them. If you do not receive a proof, you can assume this data is lost and make sure that piece of your date is reduplicated on the network. if a storage provider does not get paid for his proof, he can assume the user does not longer require him to store the data allowing him to delete this shard. 

- There are still some issues though: 
  - hosts take a risk when storing data, as they are never sure they will be paid. in the most extreme case, a malicious host might store a lot of data on the network, but never pay the first proof of storage, wasting this storage space. This effectively executes a denial of service attack on the network.
  - hosts do have no cost of not providing a proof of storage. 
  this makes the network very vulnerable to a sybil attack \ref{sybil-attack}. in which one malicious agent pretends to be several hosts at same time, at no cost. renters might think their data is secure, as it is duplicated over several hosts. However, the one malicious agent can turn off all his hosts, effectively destroying the stored data.


\fi


Storj uses a pay as you go system. From time to time, the renter asks the hosts for a proof of storage that they still hold the data. If the renter receives this proof, he pays them. If the renter does not receive the proof, his assumes this host does no longer store his data and the renter makes sure that piece is reduplicated on the network. If a host does not receive a payment for a proof of storage, he assumes the renter does no longer require him to store the data and that he can delete the renter's data. \cite{storj}

Although this solves some of the original incentive issues, there are still other issues. 

Hosts are never sure they will be paid when storing data. This means that the network is vulnerable to a denial of service attack. If a malicious agent stores a lot of data on the network without ever paying the first proof of storage, the network's storage space is being wasted, making it unavailable for honest renters.

Furthermore, hosts still have no cost of not providing a proof of storage. This makes the network more vulnerable to a Sybil attack \cite{sybil-attack}, in which a malicious agent pretends to be multiple hosts without any cost. A renter might think his data is secure as it seems duplicated over several hosts. However, the malicious agent controlling these hosts could destroy the data at any time.

\subsection{Siacoin}

\iffalse
#siacoin

- instead of pay as you go system, uses siacoin a blockchain to support their storage network. A host and a renter both sign a smart contract onto the siacoin blockchain. In this smart contract, the renter puts in his payment for the storage and the host puts in a collatoral in case he is not able to produce a proof of storage when the contract expires. Consequentially, the host is assured he will be paid by the blockchain, even when the renter is offline at the time the contract expires. And the renter is has a better assurance the host will provide the stored data as he would be rewarded the collatoral instead.

- Because the siacoin blockchain is publicly available, hosts' peformance is publicly visible. Hosts that are always providing a proof of storage when asked will be very reputable, while hosts that are less available will have a lower reputation or might even be blacklisted.
gs
- Finally, siacoin has an additional protection against a sybil attack. Although hosts have to pay a collateral if they do not provide a proof of storage, a malicious agent might still spin up an insane amount of hosts. His hosts are very likely to be picked to store data for the same renter. While it is not costless, the data duplication redundancy is not sufficient to prevent the malicious agent from effectively destroying the data of this renter. To prevent this situation, renters expect hosts to prove they are real by providing a proof-of-burn. This means that a host send coins to an address that cannot spend these coins.

\fi

Instead of a pay as you go system, Siacoin uses a blockchain to support their network. A host and a renter both sign a smart contract onto the Siacoin blockchain. In this smart contract, the renter puts in his payment for the storage and the host puts in a collateral in case he is not able to produce a proof of storage when the contract expires. Consequentially, the host is assured he will be paid by the blockchain, even when the renter is offline at the time the contract expires. The renter has a better assurance the host will provide the stored data as he would be rewarded the collateral instead.

Because the Siacoin blockchain is publicly available, hosts' performance is publicly visible. Hosts that always provide a proof of storage when asked, will be very reputable. Less available hosts will have a lower reputation or might even be blacklisted.

Finally, Siacoin has an additional protection against a Sybil attack. Although hosts have to pay a collateral if they do not provide a proof of storage, a malicious agent might still spin up an insane amount of hosts. His hosts are very likely to be picked to store data for the same renter. While it is not costless, the data duplication redundancy is not sufficient to prevent the malicious agent from effectively destroying the data of this renter. To prevent this situation, a renter expects a host to prove they are real by providing a proof-of-burn. This means that a host sends coins to an address that cannot spend these coins. \cite{siacoin}