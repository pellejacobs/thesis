\section{Timestamping a database}
\label{timestamping-db}

% add right to be forgotten reference to deloitte interview

Although blockchains have many use cases for storage of distributed data, section \ref{resource-allocation} shows that blockchain technology has some serious constraints as well. This section proposes an alternative to storage directly on the blockchain if the use case needs scalability or needs the functionality to delete records. 

Because the blocks in a blockchain are never deleted, a blockchain only grows in size. If the transactions of a blockchain consist of more than a small message, the blockchain could quickly explode in size. As a result, any use case that requires the peers in the network to store larger amounts of data cannot use a blockchain.

Some use cases also require deletion functionality. As Jean-Luc Verhelst, blockchain expert at Deloitte Belgium, explains: one of the issues of implementing blockchains inside the European Union is dealing with the European right to be forgotten (J.-L. Verhelst, personal interview, July 10, 2017). The proposed solution in section \ref{blacklist} would be illegal considering the right to be forgotten, as it is never possible to completely remove a bad client from the records. 

To evade these issues, peers in the network can decide to use a database similar to how git operates. Every peer holds its own version of the data. If a peer changes an entry, he propagates the change through the network. If a peer receiving such a change thinks this change is fine, he might decide to implement the change. If the change conflicts with another change to the same data, this peer might decide not to implement the change. Because not everyone implements the same changes, all versions of the database will gradually start to differ. 

To fix these differences, all peers will come together every now and then to manually agree on a consensus concerning the differences between the databases. Once consensus is achieved, all peers agree on a single truthful version of the data. To ensure immutability, participants then take a fingerprint of the database, digitally sign this fingerprint and post it to a public blockchain, eg. the Bitcoin blockchain.

Peers can now continue to work from this new version of the database and remove all previous data. If a peer wants to prove the distributed database contained certain data at a certain time, he only has to show that an exact copy of this version of the database has the same fingerprint as the fingerprint stored on the public blockchain at that time. As the fingerprint has been signed by all peers, all peers agreed at the time that was the version of the truth. This essentially timestamps the database.

% how does this solve scalability and right to be forgotten

This solves the scalability issue as a digital fingerprint is agnostic of the size of its subject. The size of the database does not matter anymore. Deletion functionality is available as well. Once consensus has been achieved, all previous versions of the database can be deleted. Applied to the blacklist example of section \ref{blacklist}, peers can decide to remove a client from the blacklist essentially removing him from the database used for related decisions.


% what are the disadvantages
\iffalse
disadvantages: 
- very costly consensus process: manual agreement on the data can take a long time, during the consensus process, the data cannot be changed in a production environment.
- very high latency: only the database on which there has been achieved consensus can be used to make important decisions. the constant changes from peers can be incorporated into small decisions, but a peer is never sure such a change will be incorporated in the next consensus agreement.
\fi

This solution has some disadvantages as well. First, the consensus process can be very costly. It can take a long time before all peers achieve manual agreement. Furthermore, data cannot be changed in a production environment during the consensus process, as nobody knows what the next version of the database will look like. Finally, this solution involves a very high latency. Only the latest version of the database on which consensus has been achieved can be used to make important decisions. Although the constant updates from the peers can be incorporated in a working version of the database, a peer will never be sure such a change will be incorporated into the next consensus agreement.

Because of these disadvantages, the timestamping solution should only be used if there is no other option.