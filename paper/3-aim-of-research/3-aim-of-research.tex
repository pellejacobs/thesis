\chapter{Aim of research}

\iffalse
- start with consensus and consistency in col. distr. dbs => because no central authority
- pivot into main idea: have technology replace an indispensable middleman. As discussed in \ref{subsec:examples-distributed-dbs}, blockchains solve both issues very well. 
- to apply the main idea to collaborative databases: two approaches. First a centralized collaborative database can be stored without a middleman. Second, the interaction between a nodes of a distributed collaborative database can be managed without a middleman
- For the first approach, the final chapter looks into proposals to store data in a distributed manner. Every proposal is examined on advantages, disadvantages, viability and current implementations.
- Concerning the second approach, the last chapter focuses on proposals to manage the interaction 
\fi

The central research question of this paper asks ``How can this idea to replace a middleman with technology be applied to collaborative, distributed databases?''. Section \ref{sec:distributed-dbs} shows collaborative, distributed databases face consistency and consensus issues. These issues arise because there is no central authority or trusted middleman. Sometimes some authority is created, such as in the example of git. However, section \ref{subsubsec:blockchain-as-distributed-dbs} indicates that blockchain solves both the consensus and consistency issue without the need for a trusted authority.

When applying the main idea of replacing a middleman with technology to collaborative databases, two use cases come up. The first case focuses on a centralized database that is stored without a trusted middleman. The second case examines the management of the interactions between the nodes of a distributed database without a central, guiding authority.

The first case studies centralized, collaborative databases. Centralized databases have not been heavily covered in the previous chapters as they are rather trivial. But it is, of course, possible to collaborative on a centralized database as well. An example would be a shared Dropbox folder. Several users work together on the same files, stored at a trusted third party \cite{dropbox-sharing}. The final chapter looks into proposals to store this data in a distributed way instead. It is important to note that proposals in this category do not intent to solve either consensus or consistency issue. Because the collaborative databases are centralized to begin with, this would not be possible.

The second case focuses on decentralized, collaborative databases. As often discussed before, the two main issues are consensus between the nodes and consistency between the data stored on the nodes. The final chapter examines several proposals to manage these interactions in a distributed manner without a central authority.


